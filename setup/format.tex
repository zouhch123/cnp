% Standard packages

\usepackage[english]{babel}
\usepackage{times}
\usepackage{tabularx,multirow,multicol,keystroke,subfigure,longtable}
\usepackage{setspace}

% \usepackage[utf8]{inputenc}
% \usepackage[OT1]{fontenc}

% Graphix
\usepackage{graphicx}

% Setup TikZ
\usepackage{tikz}

% 增加注释
\usepackage{silence}
\WarningsOff
\usepackage{pdfcomment}
\newcommand{\pdfnote}[1]{\marginnote{\pdfcomment[icon=note]{#1}}}
\WarningsOn

\WarningFilter{latexfont}{Font shape `TU/NotoSansCJKSC(0)/m/it'}
\WarningFilter{latexfont}{Some font shapes were not available}

% justifying
\usepackage{ragged2e}
\apptocmd{\frame}{}{\justifying}{}

% 字体
\setsansfont{Source Sans Pro}
\setmainfont{Source Serif Pro}
\setmonofont{Source Code Pro}

% 中文包
\usepackage{ctex}
\setCJKmainfont{Noto Serif CJK SC}
\setCJKsansfont{Noto Sans CJK SC}
\setCJKmonofont{Noto Sans CJK SC}

% 定制代码输出
\usepackage{minted}
% \usemintedstyle{xcode}

% 定义罗马数字
\makeatletter
\newcommand{\rmnum}[1]{\romannumeral #1}
\newcommand{\Rmnum}[1]{\expandafter\@slowromancap\romannumeral #1@}
\makeatother

% 定义破折号
\newcommand{\pozhehao}{\kern0.3ex\rule[0.8ex]{2em}{0.1ex}\kern0.3ex}

% 中文图表
\renewcommand\figurename{图}
\renewcommand\tablename{表}

% varblock
\newenvironment<>{varblock}[2][\textwidth]{
  \begin{center}
    \begin{minipage}{#1}
      \setlength{\textwidth}{#1}
      \begin{actionenv}#3
        \def\insertblocktitle{#2}
        \par
        \usebeamertemplate{block begin}}
        {\par
        \usebeamertemplate{block end}
      \end{actionenv}
    \end{minipage}
  \end{center}}

%%% End:
